%!TEX root = ../../master.tex

\chapter{rbaStrength} % (fold)
\label{cha:rbaStrength} % Insert function name

\section{Purpose} % (fold)
\label{sec:rbaStrength_purpose}
% A short description of the purpose of the function
Calculate the sound strength $G$ in dB according to ISO 3382-1.
% section purpose (end)

\section{Syntax} % (fold)
\label{sec:rbaStrength_syntax}
% How is it called in MATLAB

\texttt{G = rbaStrength(h,h10,fs)}\\
\texttt{G = rbaStrength(h,h10,fs,tInf)}
% section syntax (end)

\section{Description} % (fold)
\label{sec:rbaStrength_description}
% Detailed description of the function
The sound strength is defined as the logarithmic ratio of the squared integrated impulse response to the direct sound energy recorded in the free field at a distance of 10 m from the sound source,
\begin{equation}
	G = 10\log\frac{\int_0^{\infty} h^2(t)\; dt}{\int_0^{\infty}h^2_{10}(t)\; dt} \text{dB}
\end{equation}


% section description (end)

\section{Optional Input Arguments} % (fold)
\label{sec:rbaStrength_optional_input_arguments}
% List of optional input arguments and a short description
\begin{itemize}
	\item[-]\texttt{tInf} : The time in seconds that is greater than or equal to the point at which the decay curve has decreased by 30 dB. If \texttt{tInf} is larger than the duration of \texttt{h} it is set to \texttt{tInf = length(h)}.
\end{itemize}
% section optional_input_arguments (end)

\section{Example} % (fold)
\label{sec:rbaStrength_example}
% A short demo script using the function.
\begin{lstlisting}[style=nonumbers]
[h,fs] = wavread('ImpulseResponse.wav');
h10 = wavread('ImpulseResponseAt10m.wav');
G = rbaStrength(h,h10,fs);
\end{lstlisting}

% section example (end)

\section{See Also} % (fold)
\label{sec:rbaStrength_see_also}

% Similar or related functions

% section see_also (end)

\section{References} % (fold)
\label{sec:rbaStrength_references}

% \usepackage{chapterbib}

% section references (end)

% chapter function_name (end)