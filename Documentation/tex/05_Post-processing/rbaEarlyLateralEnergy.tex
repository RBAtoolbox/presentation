%!TEX root = ../../master.tex

\chapter{rbaEarlyLateralEnergy} % (fold)
\label{cha:rbaEarlyLateralEnergy} % Insert function name

\section{Purpose} % (fold)
\label{sec:rbaEarlyLateralEnergy_purpose}
% A short description of the purpose of the function
Calculate early late energy according to ISO-3382-1

% section purpose (end)

\section{Syntax} % (fold)
\label{sec:rbaEarlyLateralEnergy_syntax}

% How is it called in MATLAB

%\texttt{y = function_name(input, input2)}
\texttt{LF = rbaEarlyLateralEnergy(pL, p, fs)}

% section syntax (end)

\section{Algorithm} % (fold)
\label{sec:rbaEarlyLateralEnergy_algorithm}
% (OPTIONAL) Which algorithm for calculation is used.
The function uses the calculation of the early lateral energy.

\begin{equation*}
J_{LF} = \frac{\int_{0,005}^{0,080} p_L^2(t) \mathrm{d}t}{\int_{0}^{0,080} p^2(t) \mathrm{d}t}
\end{equation*}

where
\begin{itemize}
    \item[$p_L(t)$] is the instaneous sound pressure of the impulse response, measured with a figure-of-eight microphone.
    \item[$p(t)$] is the instaneous sound pressure of the impulse response, measured at the measurement point.
\end{itemize}

% section algorithm (end)

\section{Description} % (fold)
\label{sec:rbaEarlyLateralEnergy_description}
% Detailed description of the function


% section description (end)
\section{Optional Input Arguments} % (fold)
\label{sec:rbaEarlyLateralEnergy_optional_input_arguments}

% List of optional input arguments and a short description

% section optional_input_arguments (end)

\section{Example} % (fold)
\label{sec:rbaEarlyLateralEnergy_example}

% A short demo script using the function.

% section example (end)

\section{See Also} % (fold)
\label{sec:rbaEarlyLateralEnergy_see_also}

% Similar or related functions
\texttt{rbaLateLateralEnergy}
% section see_also (end)

\section{References} % (fold)
\label{sec:rbaEarlyLateralEnergy_references}

% \usepackage{chapterbib}

% section references (end)

% chapter function_name (end)