%!TEX root = ../../master.tex

\chapter{rbaMeasurement} % (fold)
\label{cha:rbaMeasurement} % Insert function name

\section{Purpose} % (fold)
\label{sec:rbaMeasurement_purpose}
% A short description of the purpose of the function
Perform measurement.
This function requires PsychToolbox \url{http://psychtoolbox.org/HomePage}

% section purpose (end)

\section{Syntax} % (fold)
\label{sec:rbaMeasurement_syntax}

\texttt{y = rbaMeasurement(signal, fs, N, estimatedRT)}

% section syntax (end)

\section{Description} % (fold)
\label{sec:rbaMeasurement_description}

Connections are established to the default input and output of the computers operating system, and a system or room response is recorded. The input \texttt{signal} is used to excite the room. The measurements are averaged over \texttt{N} recordings.\\
Each excitation is followed by a quiet time of \texttt{estimatedRT} seconds. This method is called a transient measurement, and while it takes slightly longer, than with a sound field build-up, it is very reliable and produces robust results. \texttt{estimatedRT} should be an estimate of the reverberation time.

% section description (end)

\section{Example} % (fold)
\label{sec:rbaMeasurement_example}

\begin{lstlisting}
% Generate logarithmic sine sweep
fs = 48000;         % Sampling frequency
f1 = 1000;           % Lower frequency
f2 = 8000;          % Upper frequency
length_sig = 5;     % Duration of sweep in seconds

[sweep,t] = rbaGenerateSignal('logsin',fs,f1,f2,length_sig);

% Start measurement
RT = 1;             % Estimated reverberation time of room
N = 4;              % number of sweeps to average over
transient = 0;
y = rbaMeasurement(sweep,fs,N,RT,transient);
% Compute impulse response
h = sweepdeconv(sweep,y,f1,f2,fs);

figure('Name','Spectrogram','Position',[0 400 400 400])
specgram(y,min(256,length(y)),fs)
figure('Name','Impulse response','Position',[420 400 400 400])
plot(h)
\end{lstlisting}

% section example (end)

\section{See Also} % (fold)
\label{sec:rbaMeasurement_see_also}

\texttt{\nameref{cha:rbaGenerateSignal}}

% section see_also (end)

% chapter function_name (end)