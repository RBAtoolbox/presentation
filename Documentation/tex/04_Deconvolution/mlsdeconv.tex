%!TEX root = ../../master.tex

\chapter{mlsdeconv} % (fold)
\label{cha:mlsdeconv} % Insert function name

\section{Purpose} % (fold)
\label{sec:mlsdeconv_purpose}
% A short description of the purpose of the function
Computes the circular crosscorrelation function between signals \texttt{x} and \texttt{y}.

% section purpose (end)

\section{Syntax} % (fold)
\label{sec:mlsdeconv_syntax}
% How is it called in MATLAB

\texttt{[h,t] = mlsdeconv(x,y,fs[,'AmpCorr'])}

% section syntax (end)

\section{Description} % (fold)
\label{sec:mlsdeconv_description}
% Detailed description of the function
The circular crosscorrelation function between input and output signals is a good approximation of the impulse response of a system. Given \texttt{y} is the output signal of a \emph{linear} system, and \texttt{x} is the output, this is used to do real-time analysis on mls measurements over long periods of time.

\texttt{mlsdeconv} uses the utility function \texttt{flip} to flip the elements of \texttt{x}.
% section description (end)

\section{Optional Input Arguments} % (fold)
\label{sec:mlsdeconv_optional_input_arguments}
% List of optional input arguments and a short description

\begin{itemize}
	\item[-] \texttt{'AmpCorr'}: The use of this string enables the option of correcting the impulse response with a scale factor.
\end{itemize}

% section optional_input_arguments (end)

\section{See Also} % (fold)
\label{sec:mlsdeconv_see_also}
% Similar or related functions
\texttt{\nameref{cha:sweepdeconv}}, \quad \texttt{\nameref{cha:rbaLinearConv}}, \quad \texttt{\nameref{cha:rbaCircularConv}}.

% section see_also (end)

\section{References} % (fold)
\label{sec:mlsdeconv_references}

% \usepackage{chapterbib}

% section references (end)

% chapter function_name (end)