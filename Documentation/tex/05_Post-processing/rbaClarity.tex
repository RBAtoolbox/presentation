%!TEX root = ../../master.tex

\chapter{rbaClarity} % (fold)
\label{cha:rbaClarity} % Insert function name

\section{Purpose} % (fold)
\label{sec:rbaClarity_purpose}

% A short description of the purpose of the function

Calculates Clarity (e.g $C_{30}$ or $C_{80}$) from impulse response, according to ISO 3382-1:2009 %cite

% section purpose (end)

\section{Syntax} % (fold)
\label{sec:rbaClarity_syntax}

% How is it called in MATLAB

\texttt{C = rbaClarity(h,fs,time)}

% section syntax (end)

\section{Algorithm} % (fold)
\label{sec:rbaClarity_algorithm}

% (OPTIONAL) Which algorithm for calculation is used.

The clarity parameter is calculated according to ISO-3382-1

\begin{equation}
    C_{t_e} = 10 \log \frac{\int_0^{t_e}h^2(t)\mathrm{d}t}{\int_{t_e}^\infty h^2(t)\mathrm{d}t} \SI{}{dB}
\end{equation}
where $t_e$ is the input time constant, and $h(t)$ is the room implulse response of interest.

% section algorithm (end)

\section{See Also} % (fold)
\label{sec:rbaClarity_see_also}

% Similar or related functions
\texttt{\nameref{cha:rbaCentreTime}},\quad \texttt{\nameref{cha:rbaDefinition}},\quad \texttt{\nameref{cha:rbaEDT}},\quad \texttt{\nameref{cha:rbaReverberationTime}}.
% section see_also (end)

\section{References} % (fold)
\label{sec:rbaClarity_references}

% \usepackage{chapterbib}
ISO 3382-1:2009

% section references (end)

% chapter function_name (end)