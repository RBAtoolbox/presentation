%!TEX root = ../../master.tex

\chapter{sweepdeconv} % (fold)
\label{cha:sweepdeconv} % Insert function name

\section{Purpose} % (fold)
\label{sec:sweepdeconv_purpose}
% A short description of the purpose of the function
Performs a linear deconvolution of input and output of a system. This can be a room measurement, being correlated with the excitation signal, which ends up in the room impulse response.
% section purpose (end)

\section{Syntax} % (fold)
\label{sec:sweepdeconv_syntax}

% How is it called in MATLAB
\texttt{[h,t] = sweepdeconv(x,y,f1,f2,fs)}

% section syntax (end)

\section{Description} % (fold)
\label{sec:sweepdeconv_description}

A deconvolution is basically a cross correlation between measurement and excitation signal. When doing measurements with logarithmic sweeps, the linear deconvolution should be used, since the sweep response will have a decay, that longens the signal.

% section description (end)

\section{See Also} % (fold)
\label{sec:sweepdeconv_see_also}

% Similar or related functions
\texttt{mlsdeconv}, \texttt{rbaLinearConv}, \texttt{rbaCircularConv}
% section see_also (end)
\underline{}

% chapter function_name (end)