%!TEX root = ../../master.tex

\chapter{rbaBassRatio} % (fold)
\label{cha:rbaBassRatio} % Insert function name

\section{Purpose} % (fold)
\label{sec:rbaBassRatio_purpose}

Computes the Bass Ratio of a given room impulse response.
% section purpose (end)

\section{Syntax} % (fold)
\label{sec:rbaBassRatio_syntax}

% How is it called in MATLAB

\texttt{BR = rbaBassRatio(h,fs)}

% section syntax (end)

\section{Algorithm} % (fold)
\label{sec:rbaBassRatio_algorithm}

% (OPTIONAL) Which algorithm for calculation is used.
The Bass Ratio is calculated as the ratio between reverberation times of low and high frequency octave bands. Note: 250~Hz band omitted. 
\begin{equation}
	\text{BR} = \frac{T_{63}+T_{125}}{T_{500}+T_{1000}+T_{2000}}
\end{equation}
% section algorithm (end)

\section{See Also} % (fold)
\label{sec:rbaBassRatio_see_also}

% Similar or related functions

\texttt{rbaCentreTime}, \texttt{rbaClarity}, \texttt{rbaDefinition}, \texttt{rbaEDT}, \texttt{rbaReverberationTime}, \texttt{rbaStrength}

% section see_also (end)

\section{References} % (fold)
\label{sec:rbaBassRatio_references}

Niels Werner Adelman-Larsen, Eric R. Thompson and Anders Christian Gade
Suitable reverberation times for halls for rock and pop music\\
Journal of the Acoustical Society of America, 2010, Vol. 127, nr. 1, pp. 247-255


% section references (end)

% chapter function_name (end)