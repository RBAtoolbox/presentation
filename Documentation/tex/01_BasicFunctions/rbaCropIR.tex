%!TEX root = ../../master.tex

\chapter{rbaCropIR} % (fold)
\label{cha:rbaCropIR} % Insert function name

\section{Purpose} % (fold)
\label{sec:rbaCropIR_purpose}
% A short description of the purpose of the function
Crop impulse response to remove onset and noise.

% section purpose (end)

\section{Syntax} % (fold)
\label{sec:rbaCropIR_syntax}

% How is it called in MATLAB

\texttt{[hCrop,t,Onset,knee] = rbaCropIR(h,fs)}\\
\texttt{[hCrop,t,Onset,knee] = rbaCropIR(h,fs,'onset')}\\
\texttt{[hCrop,t,Onset,knee] = rbaCropIR(h,fs,'tight')}\\
% section syntax (end)

\section{Description} % (fold)
\label{sec:rbaCropIR_description}
% Detailed description of the function
Impulse responses need to be cropped in order to produce proper results when computing parameters such as \texttt{\nameref{cha:rbaReverberationTime}}. ISO 3382-1 provides a procedure to determine the onset of a broadband impulse response: The onset is the point where the impulse response rises significantly above the background noise but is more than 20 dB below the maximum.

The knee-point where the decay meets the noise floor can be determined by Lundeby's method.
% section description (end)

\section{Optional Input Arguments} % (fold)
\label{sec:rbaCropIR_optional_input_arguments}
% List of optional input arguments and a short description
\begin{itemize}
	\item[-] \texttt{'onset'}: Crop only onset
	\item[-] \texttt{'tight'}: Crop onset and noise by Lundeby's method
\end{itemize}

% section optional_input_arguments (end)

\section{Example} % (fold)
\label{sec:rbaCropIR_example}
% A short demo script using the function.

\begin{lstlisting}[style=nonumbers]
[h,fs] = wavread('ImpulseResponse.wav');
[hCrop,t] = rbaCropIR(h,fs,'onset');
\end{lstlisting}

% section example (end)

\section{See Also} % (fold)
\label{sec:rbaCropIR_see_also}
% Similar or related functions

\texttt{\nameref{cha:rbaLundeby}},\quad \texttt{\nameref{cha:rbaReverberationTime}}

% section see_also (end)

\section{References} % (fold)
\label{sec:rbaCropIR_references}
ISO 3382-1.
% \usepackage{chapterbib}

% section references (end)

% chapter function_name (end)