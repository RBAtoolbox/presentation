%!TEX root = ../../master.tex

\chapter{rbaCrossCorr} % (fold)
\label{cha:rbaCrossCorr} % Insert function name

\section{Purpose} % (fold)
\label{sec:rbaCrossCorr_purpose}

% A short description of the purpose of the function
Calculate cross correlation


% section purpose (end)

\section{Syntax} % (fold)
\label{sec:rbaCrossCorr_syntax}

% How is it called in MATLAB

\texttt{[c, lags] = rbaCrossCorr(f, g)}

% section syntax (end)

\section{Algorithm} % (fold)
\label{sec:rbaCrossCorr_algorithm}

% (OPTIONAL) Which algorithm for calculation is used.
The definition of cross correlation is
\begin{equation*}
    (f \star g) [n] = \sum_{m=-\infty}^{\infty} f^*[m]g[n+m]
\end{equation*}

The function makes use of the following: That a convolution in time, is a multiplication in frequency.

\begin{equation*}
     \mathcal{F}\{ f \star g\} = \mathcal{F}\{ f\} \mathcal{F}\{ g \}
\end{equation*}

% section algorithm (end)

\section{Description} % (fold)
\label{sec:rbaCrossCorr_description}
    Since Matlabs built-in cross correlation sends calculations to
    \texttt{fftfilt}, when \texttt{length(f)>=10*length(h)}, and \texttt{fftfilt} does not support
    longer signals than $2^{20}=1048576$, the need for a more robust
    implementation was born.
    This function is written with the purpose of determining the exact
    position of a single sweep response in a room acoustic measurement,
    which consists of the response of several sweeps.
    The cross correlation searches for patterns, similar to h, in
    signal f. It is closely related to the convolution function.\\
    \\
    \texttt{c} is the result of the cross-correlation for $n$
    \texttt{lags} is a vector containing the lags $n$


% section description (end)

% \section{Optional Input Arguments} % (fold)
% \label{sec:rbaCrossCorr_optional_input_arguments}

% % List of optional input arguments and a short description

% % section optional_input_arguments (end)


\section{See Also} % (fold)
\label{sec:rbaCrossCorr_see_also}
% Similar or related functions
\texttt{\nameref{cha:rbaLinearConv}}, \quad \texttt{\nameref{cha:rbaCircularConv}}.
% section see_also (end)

% chapter function_name (end)