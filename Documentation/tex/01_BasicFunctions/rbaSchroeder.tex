%!TEX root = ../../master.tex

\chapter{rbaSchroeder} % (fold)
\label{cha:rbaSchroeder} % Insert function name

\section{Purpose} % (fold)
\label{sec:rbaSchroeder_purpose}
% A short description of the purpose of the function
Compute decay curves from Schröders backward integration method.

% section purpose (end)

\section{Syntax} % (fold)
\label{sec:rbaSchroeder_syntax}
% How is it called in MATLAB
\texttt{R = rbaSchroeder(h,fs,flag)}\\
\texttt{R = rbaSchroeder(h,fs,flag,'Lundeby')}\\
\texttt{R = rbaSchroeder(h,fs,flag,knee)}
\begin{itemize}
	\item[-] \texttt{flag = \{1,0\}}: Enables noise compensation (tested but not implemented yet).
\end{itemize}

% section syntax (end)

\section{Algorithm} % (fold)
\label{sec:rbaSchroeder_algorithm}

% (OPTIONAL) Which algorithm for calculation is used.

% section algorithm (end)

\section{Description} % (fold)
\label{sec:rbaSchroeder_description}
% Detailed description of the function
The decay curve
\begin{align}
	R(t) &= 10\cdot\log_{10}\left(\int_{\infty}^{t}h^2(t)\id(-\tau)\right)\\
	R(t) &= 10\cdot\log_{10}\left(\int_{t_1}^t h^2(t)\id(-\tau)+\int_{t_1}^{\infty} h_{exp}^2(t)\id(-\tau)	\right)
\end{align}
The latter integral is an optional correction factor to minimize the influece of noise (enabled by setting \texttt{flag = 1}). It is calculted under the assumption of exponential decay of energy with the same slope as $h^2(t)$ between $t_0$ and $t_1$, where $t_0$ is the time corresponding to a level 10 dB higher than the level at $t_1$. The slope is calculated from the time-averaged energy decay and a linear least-squares fit is found to $y(t) = 10\cdot\log_{10}\left(h_{exp}^2(t)\right)$, with
\begin{equation}
	h_{exp}^2(t) = E_0\exp^{At}.
\end{equation}
The coefficients of the linear fit $y_{fit}(t) = at+b$ is connected to the exponential coefficients $E_0$ and $A$, by
\begin{equation}
	E_0 = 10^{b/10}\; ,\quad A = \ln\left(10^{a/10}\right)
\end{equation}

% section description (end)

\section{Optional Input Arguments} % (fold)
\label{sec:rbaSchroeder_optional_input_arguments}
% List of optional input arguments and a short description
\begin{itemize}
	\item[-] \texttt{'Lundeby'}: Use Lundeby's method to calculate the knee-point where the exponential decay meets the noise floor (default).
	 \item[-] \texttt{knee}: Index value of the knee-point found by e.g. a graphical approach.
\end{itemize}
% section optional_input_arguments (end)

\section{Example} % (fold)
\label{sec:rbaSchroeder_example}

% A short demo script using the function.

% section example (end)

\section{See Also} % (fold)
\label{sec:rbaSchroeder_see_also}
% Similar or related functions
\texttt{\nameref{cha:rbaLundeby}},\quad \texttt{\nameref{cha:rbaCropIR}}
% section see_also (end)

\section{References} % (fold)
\label{sec:rbaSchroeder_references}

% \usepackage{chapterbib}

% section references (end)

% chapter function_name (end)