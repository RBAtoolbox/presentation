%!TEX root = ../../master.tex

\chapter{rbaGetFreqs} % (fold)
\label{cha:rbaGetFreqs} % Insert function name

\section{Purpose} % (fold)
\label{sec:rbaGetFreqs_purpose}
% A short description of the purpose of the function
Return standardized octave band center frequencies vector for plotting purposes in acoustics.

% section purpose (end)

\section{Syntax} % (fold)
\label{sec:rbaGetFreqs_syntax}

% How is it called in MATLAB

%\texttt{y = function_name(input, input2)}
\texttt{freqs = rbaGetFreqs(cfmin,cfmax,BandsPerOctave)}
% section syntax (end)

% \section{Algorithm} % (fold)
% \label{sec:rbaGetFreqs_algorithm}

% % (OPTIONAL) Which algorithm for calculation is used.

% % section algorithm (end)

\section{Description} % (fold)
\label{sec:rbaGetFreqs_description}
% Detailed description of the function
\texttt{cfmin} and \texttt{cfmax} determine the range and \texttt{BandsPerOctave} should be \texttt{1} or \texttt{3}.

% section description (end)

% \section{Optional Input Arguments} % (fold)
% \label{sec:rbaGetFreqs_optional_input_arguments}

% % List of optional input arguments and a short description

% % section optional_input_arguments (end)

\section{Example} % (fold)
\label{sec:rbaGetFreqs_example}
% A short demo script using the function.
\begin{lstlisting}
cfmin = 63;
cfmax = 4e3;
BandsPerOctave = 1;
f = rbaGetFreqs(cfmin, cfmax);
% Plot reverberation time for each octave-band
plot(f, RT)
\end{lstlisting}
% section example (end)

\section{See Also} % (fold)
\label{sec:rbaGetFreqs_see_also}
% Similar or related functions
\texttt{\nameref{cha:rbaFilterBank}},\quad \texttt{\nameref{cha:rbaIR2OctaveBands}}.
% section see_also (end)

% \section{References} % (fold)
% \label{sec:rbaGetFreqs_references}

% % \usepackage{chapterbib}

% % section references (end)

% chapter function_name (end)