%!TEX root = ../../master.tex

\chapter{rbaIR2OctaveBands} % (fold)
\label{cha:rbaIR2OctaveBands} % Insert function name

\section{Purpose} % (fold)
\label{sec:rbaIR2OctaveBands_purpose}

% A short description of the purpose of the function
Filter impulse response into 1-octave or $1/3$-octave bands

% section purpose (end)

\section{Syntax} % (fold)
\label{sec:rbaIR2OctaveBands_syntax}

% How is it called in MATLAB

%\texttt{y = function_name(input, input2)}
  \texttt{H = rbaIR2OctaveBands(h, fs, cfmin, cfmax)} \\
  \texttt{H = rbaIR2OctaveBands(h, fs, cfmin, cfmax, bandsPerOctave)} \\
  \texttt{H = rbaIR2OctaveBands(h, fs, cfmin, cfmax, bandsPerOctave, reverse)}
% section syntax (end)

% \section{Algorithm} % (fold)
% \label{sec:rbaIR2OctaveBands_algorithm}

% % (OPTIONAL) Which algorithm for calculation is used.

% % section algorithm (end)

\section{Description} % (fold)
\label{sec:rbaIR2OctaveBands_description}
% Detailed description of the function
The room acoustic parameters are often computed in octave or third-octave frequency bands. This function will return a matrix \texttt{H} containing the filtered impulse response for each octave bands with center frequencies in the range \texttt{cfmin;cfmax]}.
\\
The function uses fourth-order Butterworth filters with cut-off frequencies at the edges of the band. This function requires the \emph{Signal Processing Toolbox} included with most versions of \emph{MATLAB}, but also available from the \emph{MathWorks} website.\\
The function includes the option to filter the time-reversed impulse response. This can be done to minimize the influence of the filter on the decay-curve as described in \cite{} and \cite{}.



% section description (end)

% \section{Optional Input Arguments} % (fold)
% \label{sec:rbaIR2OctaveBands_optional_input_arguments}
% % List of optional input arguments and a short description

% % section optional_input_arguments (end)

\section{Example} % (fold)
\label{sec:rbaIR2OctaveBands_example}
% A short demo script using the function.
\begin{lstlisting}
[h, fs] = wavread('measured_impulse_response');
cfmin = 63;
cfmax = 4000;
H = rbaIR2OctaveBands(h, fs, cfmin, cfmax);
\end{lstlisting}


% section example (end)

\section{See Also} % (fold)
\label{sec:rbaIR2OctaveBands_see_also}
% Similar or related functions
\texttt{\nameref{cha:rbaFilterBank}},\quad \texttt{\nameref{cha:rbaGetFreqs}}%,\quad \texttt{\nameref{cha:rbaGetFreqs}}

% section see_also (end)

\section{References} % (fold)
\label{sec:rbaIR2OctaveBands_references}
Jacobsen, F. (..). A Note on Acoustic Decay Measurement, 1–8.\\
Jacobsen, F. (..). Time-reversed decay Measurement. (F. Jacobsen, Ed.), 1–4.\\
% \usepackage{chapterbib}

% section references (end)

% chapter function_name (end)