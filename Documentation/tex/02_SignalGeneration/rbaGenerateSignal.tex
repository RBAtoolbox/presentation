%!TEX root = ../../master.tex

\chapter{rbaGenerateSignal} % (fold)
\label{cha:rbaGenerateSignal} % Insert function name

\section{Purpose} % (fold)
\label{sec:rbaGenerateSignal_purpose}
% A short description of the purpose of the function
Generate measurement signal.


% section purpose (end)

\section{Syntax} % (fold)
\label{sec:rbaGenerateSignal_syntax}
% How is it called in MATLAB

%\texttt{y = function_name(input, input2)}
\texttt{[y,t] = rbaGenerateSignal('logsin',fs,f1,f2,length\_sig[,zero\_pad,amp,phase])} \\
\texttt{[y,t] = rbaGenerateSignal('linsin',fs,f1,f2,length\_sig[,zero\_pad,amp,phase])} \\
\texttt{[y,t] = rbaGenerateSignal('sin',fs,f0,length\_sig[,amp,phase])} \\
\texttt{[y,t] = rbaGenerateSignal('mls',fs, n[, amp, zero\_pad, offset])} \\

% section syntax (end)

% \section{Algorithm} % (fold)
% \label{sec:rbaGenerateSignal_algorithm}

% % (OPTIONAL) Which algorithm for calculation is used.

% % section algorithm (end)

\section{Description} % (fold)
\label{sec:rbaGenerateSignal_description}
% Detailed description of the function
\texttt{rbaGenerateSignal} is used to generate the measurement signal. The available types of signal are
\begin{itemize}
    \item Logarithmic sine sweep
    \item Linear sine sweep
    \item Single sine tone
    \item Maximum-length sequence
\end{itemize}

Each signal type requires different input arguments. There is no time-window applied to the generated signal. If af window is needed, e.g. for real world measurements using sweep signal, use \texttt{sweepwin}.\\

When applying a time-window using \texttt{sweepwin} it is necesarry to generate a sweep with a wider frequency range than the desired measurement range.

% section description (end)

\section{Optional Input Arguments} % (fold)
\label{sec:rbaGenerateSignal_optional_input_arguments}

% List of optional input arguments and a short description
\begin{itemize}
    \item[-] \texttt{zero\_pad}: duration in seconds, of silence inserted after the generated signal.
    \item[-] \texttt{amp}: the default amplitude of each signal is $1$. Use this argument if another is needed.
    \item[-] (Sweep)\texttt{phase}: Initial phase in radians of the signal. Default is $0$.
    \item[-] (MLS)\texttt{offset}: Needed offset in MLS sequence.
\end{itemize}


% section optional_input_arguments (end)

\section{Example} % (fold)
\label{sec:rbaGenerateSignal_example}

% A short demo script using the function.

\begin{lstlisting}
% Generate logarithmic sine sweep from 100 Hz to 10 kHz of length 5 s
sig_type = 'logsin';
fs = 48e3;
f1 = 100;
f2 = 10e3;
length_sig = 5;
[y, t] = rbaGenerateSignal(sig_type, fs, f1, f2, length_sig);
\end{lstlisting}

% section example (end)

\section{See Also} % (fold)
\label{sec:rbaGenerateSignal_see_also}
% Similar or related functions
\texttt{\nameref{cha:rbaMeasurement}}
%\texttt{\nameref{cha:sweepwin}}
% section see_also (end)

% \section{References} % (fold)
% \label{sec:rbaGenerateSignal_references}
% % \usepackage{chapterbib}


% % section references (end)

% chapter function_name (end)