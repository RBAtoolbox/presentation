%!TEX root = ../../master.tex

\chapter{rbaFilterBank} % (fold)
\label{cha:rbaFilterBank} % Insert function name

\section{Purpose} % (fold)
\label{sec:rbaFilterBank_purpose}
Calculate octave or 3rd-octave band filters according to the ANSI S1.11-2004. Note: The filters are only trustworthy for frequencies at and above 63 Hz at this time.
% A short description of the purpose of the function

% section purpose (end)

\section{Syntax} % (fold)
\label{sec:rbaFilterBank_syntax}

% How is it called in MATLAB
%\texttt{y = function_name(input, input2)}
\texttt{[B,A] = rbaFilterBank(BandsPerOctave,fs,cfmin,cfmax)}

% section syntax (end)

% \section{Algorithm} % (fold)
% \label{sec:rbaFilterBank_algorithm}
% % (OPTIONAL) Which algorithm for calculation is used.

% % section algorithm (end)

\section{Description} % (fold)
\label{sec:rbaFilterBank_description}
% Detailed description of the function
Calculates filter coefficients for octave or $1/3$-octave 3rd-order Butterworth bandpass filters. The function is used by \texttt{\nameref{cha:rbaIR2OctaveBands}} to filter impulse response.\\
\\
Generally \texttt{\nameref{cha:rbaIR2OctaveBands}} should be used to filter impulse response.


% section description (end)

% \section{Optional Input Arguments} % (fold)
% \label{sec:rbaFilterBank_optional_input_arguments}

% % List of optional input arguments and a short description

% % section optional_input_arguments (end)

% \section{Example} % (fold)
% \label{sec:rbaFilterBank_example}

% % A short demo script using the function.

% % section example (end)

\section{See Also} % (fold)
\label{sec:rbaFilterBank_see_also}
% Similar or related functions
\texttt{\nameref{cha:rbaIR2OctaveBands}}, \quad \texttt{\nameref{cha:rbaGetFreqs}}.

% section see_also (end)

\section{References} % (fold)
\label{sec:rbaFilterBank_references}

% \usepackage{chapterbib}

% section references (end)

% chapter function_name (end)