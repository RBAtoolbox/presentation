%!TEX root = ../../master.tex

\chapter{rbaCircularConv} % (fold)
\label{cha:rbaCircularConv} % Insert function name

\section{Purpose} % (fold)
\label{sec:rbaCircularConv_purpose}
A faster replacement of the built-in function for circular convolution.

% A short description of the purpose of the function

% section purpose (end)

\section{Syntax} % (fold)
\label{sec:rbaCircularConv_syntax}
% How is it called in MATLAB
\texttt{y = rbaCircularConv(f,h)}

% section synopsis (end)

\section{Algorithm} % (fold)
\label{sec:rbaCircularConv_algorithm}
% (OPTIONAL) Which algorithm for calculation is used.
The function uses that a convolution in the time domain is equal to a multiplication in the frequency domain.
\begin{equation*}
\mathcal{F}\{h_1 * h_2\} =  \mathcal{F}\{h_1\}\mathcal{F}\{h_2\}
\end{equation*}
% section algorithm (end)

\section{Description} % (fold)
\label{sec:rbaCircularConv_description}
% Detailed description of the function
This function does convolution faster than the built-in matlab function \texttt{conv}. It can be used to convolve periodic signals, when processing of infinite length (i.e undefined length).\\

The output of the function is not normalized.


% section description (end)

% \section{Optional Input Arguments} % (fold)
% \label{sec:rbaCircularConv_optional_input_arguments}

% % List of optional input arguments and a short description

% % section optional_input_arguments (end)

\section{See Also} % (fold)
\label{sec:rbaCircularConv_see_also}
% Similar or related functions
\texttt{rbaLinearConv}

% section see_also (end)

% \section{References} % (fold)
% \label{sec:rbaConv_references}

% % section references (end)

% chapter function_name (end)