%!TEX root = ../../master.tex

\chapter{rbaReverberationTime} % (fold)
\label{cha:rbaReverberationTime} % Insert function name

\section{Purpose} % (fold)
\label{sec:rbaReverberationTime_purpose}
% A short description of the purpose of the function
Compute $T_{20}$, $T_{30}$ and $T_{60}$ from decay curves according to ISO 3382-1.
% section purpose (end)

\section{Syntax} % (fold)
\label{sec:rbaReverberationTime_syntax}
% How is it called in MATLAB

\texttt{[RT, r2p,dynRange,stdDev] = rbaReverberationTime(R,t,'best')}\\
\texttt{[RT, r2p,dynRange,stdDev] = rbaReverberationTime(R,t,'all')}

% section syntax (end)

\section{Algorithm} % (fold)
\label{sec:rbaReverberationTime_algorithm}
% (OPTIONAL) Which algorithm for calculation is used.
The slope of the decay curves is determined from MATLAB's \texttt{polyfit} function in the intervals defined by $T_{60}$, $T_{30}$ and $T_{20}$.

The correlation coefficient squared is given by
\begin{equation}
r^2 = \frac{\sum_{i=1}^n (L_i^{\text{fit}}-\bar{L}^{\text{data}})^2}{\sum_{i=1}^n (L_i^{\text{data}}-\bar{L}^{\text{data}})^2}
\end{equation}
where $\bar{L}^{\text{data}} = \frac{1}{n}\sum_{i=1}^{n}L_i^{data}$
% section algorithm (end)

\section{Description} % (fold)
\label{sec:rbaReverberationTime_description}
% Detailed description of the function
Reverberation time is defined in ISO 3382-1 as the duration required for the space-averaged sound energy density in an enclosure to decrease by 60 dB after the source emission has stopped. This is calculated from the slope of decay curves by reverse-time integration of squared impulse responses, also known as Backward Schröder integration.
The slope is determined from a linear regression based on the available dynamic range.

\texttt{r2p} is the goodness-of-fit given by the correlation coefficient $1000\cdot(1-r^2)$.
A correlation coefficient $1000\cdot(1-r^2) = 0$ corresponds to a perfect correlation between data and fit. The decay curve is non-linear for $1000\cdot(1-r^2) > 10$ according to ISO 3382-1.

\texttt{dynRange} is the dynamic range of the decay curve in dB. If \texttt{R} is a matrix then \texttt{dynRange} is given for each column in \texttt{R}.

\texttt{stdDev} is the standard deviation of the difference between the linear regression line and the decay curve.
% section description (end)

\section{Optional Input Arguments} % (fold)
\label{sec:rbaReverberationTime_optional_input_arguments}
% List of optional input arguments and a short description
\begin{itemize}
	\item[-] \texttt{'best'}: Output only the reverberation time ($T_{60}$,$T_{30}$,$T_{20}$) with the largest dynamic range.
	\item[-] \texttt{'all'}: Output all available $T_{60}$,$T_{30}$,$T_{20}$ based on the dynamic range. The output is given in a matrix or vector based on the input.
\end{itemize}
% section optional_input_arguments (end)

\section{Example} % (fold)
\label{sec:rbaReverberationTime_example}
% A short demo script using the function.

\begin{lstlisting}[style=nonumbers]
[h,fs] = wavread('ImpulseResponse.wav');
t = 0:1/fs:length(h)/fs-1/fs;		% Construct time vector
H = rbaIR2octBands(h,fs,63,8000,0);	% Filter h in octave bands from 63 Hz to 8000 Hz (without reverse filtering)
R = rbaSchroeder(h,fs);		% Compute decay curves
[RT, r2p,dynRange,stdDev] = rbaReverberationTime(R,t,'all')
\end{lstlisting}
If a dynamic range of 65 dB is available then $T_{60}$, $T_{30}$ and
$T_{20}$ is computed. For instance,
\begin{lstlisting}[style=nonumbers]
RT =
	5.196 7.447 7.376 5.968 4.976 4.308 3.344 2.349 1.414 0.849
	5.126 7.147 7.476 5.948 4.956 4.368 3.144 2.249 1.214 0.949
	5.296 6.447 7.676 5.948 4.476 4.208 3.944 2.149 1.814 0.747
\end{lstlisting}

% section example (end)

\section{See Also} % (fold)
\label{sec:rbaReverberationTime_see_also}
% Similar or related functions
\texttt{\nameref{cha:rbaCentreTime}}, \quad \texttt{\nameref{cha:rbaClarity}}, \quad \texttt{\nameref{cha:rbaDefinition}}, \quad \texttt{\nameref{cha:rbaEDT}}.
% section see_also (end)

\section{References} % (fold)
\label{sec:rbaReverberationTime_references}
ISO 3382-1:2009
% \usepackage{chapterbib}

% section references (end)

% chapter function_name (end)